\documentclass[french,12pt]{article}
\usepackage[T1]{fontenc}
\usepackage[utf8]{inputenc}
\usepackage{lmodern}
\usepackage{indentfirst}
\usepackage[bookmarks, colorlinks=true, linkcolor=black, citecolor=black, urlcolor=blue]{hyperref}
\usepackage{amsmath}
\usepackage{hyperref}
\usepackage{hyphenat}
\usepackage[a4paper]{geometry}
\usepackage[magyar]{babel}
\author{Endreffy Zsolt}
\title{Near Field Communication}

\begin{document}

\maketitle

\pagebreak

\tableofcontents

\pagebreak

\section{Bevezetés}
Napjainkban egyre többször lehet találkozni a legkülönfélébb alkalmazási 
területeken a \textit{Near Field Communicationnel}, avagy az \textit{NFC}-vel.
De mi is ez a technológia, mi a története és hogyan is működik?

\section{A közeltér-távoltér különbségéről}

\subsection{A régiók rövid összefoglalása}
Ahogy a közeltér neve is sugallja, elsősorban az antennához közeli távolságban
beszélhetünk róla, míg távoltér esetén értelemszerűen a messzebb eső 
távolságokban. A kettő határa azonban csak homályosan van definiálva, amely
ráadásul még a forrás által sugárzott domináns hullámhossztól is függ.

A távoltérben a sugárzás elektromos és mágneses térereje a 
távolság négyzetével fordítottan arányos, 
míg a közeltérben még ennél is gyorsabban csökken, a távolság négyzetével illetve 
harmadik hatványával fordítottan arányosan.
Míg a távoltérben az elnyelődés nem hat vissza az adóra, addig a közeltérben az 
elnyelődés megváltoztatja az adó terhelését. Ennek a jelenségnek egy egyszerű
modellje a például transzformátorokban megfigyelhető mágneses indukció.

Az elektromágneses mezőben az elektromos és mágneses mező a távoltér esetén kapcsolatban
állnak és a kettejük intenzitásának arányát a hullámimpedancia adja meg. 
Ezzel ellentétben a közeltérben a két mező egymástól függetlenül létezhet, és
az egyik teljesen el tudja nyomni a másikat.

Egy antennában a pozitív és negatív töltések nem hagyhatják el az antennát és a 
gerjesztő jel által elválasztva, annak megfelelően oszcilláló elektromos dipólust
alkotnak, ami mind a közeltérben, mind a távoltérben hatást fejt ki. Az antennák
többnyire arra lettek kitalálva, hogy nagy távolságú vezetéknélküli kommunikációt
lehessen velük folytatni, és ezeknek az antennáknak ez a régió a normális működési
tartományuk. Természetesen a közeltér antennákat is egyre gyakrabban használjuk
adatátvitelre, hiszen az \textit{NFC} is ennek a képviselője.

A kettő régió között van az átmeneti zóna, amely az antenna geometriájától és a 
hullámhossz hosszától függ.

\subsection{Definíciók}



\section[NFC]{NFC, avagy Near Field Communication}


\section{Az NFC és egyéb vezetéknélküli technológiák}

\end{document}
