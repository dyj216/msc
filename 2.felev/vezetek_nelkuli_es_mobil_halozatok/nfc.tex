\documentclass[12pt]{article}
\usepackage[T1]{fontenc}
\usepackage[utf8]{inputenc}
\usepackage{lmodern}
\usepackage{indentfirst}
\usepackage[bookmarks, colorlinks=true, linkcolor=black, citecolor=black, urlcolor=blue]{hyperref}
\usepackage{amsmath}
\usepackage{hyperref}
\usepackage{hyphenat}
\usepackage[a4paper]{geometry}
\usepackage[magyar]{babel}
\author{Endreffy Zsolt}
\title{Near Field Communication}

\begin{document}

\maketitle

\pagebreak

\tableofcontents

\pagebreak

\section{Bevezetés}
Napjainkban egyre többször lehet találkozni a legkülönfélébb alkalmazási 
területeken a \emph{Field Communicationnel}, avagy az \emph{NFC}-vel.
Főként mobiltelefonokba beépítve nyer egyre nagyobb teret ez a technológia,
hiszen itt tudják a szolgáltatók a technológiát (fizetős) szolgáltatásokkal
összekötni.
De mi is ez a technológia, és hogyan is működik?

\section{A közeltér-távoltér különbségéről}

\subsection{A régiók rövid összefoglalása}
Ahogy a közeltér neve is sugallja, elsősorban az antennához közeli távolságban
beszélhetünk róla, míg távoltér esetén értelemszerűen a messzebb eső 
távolságokban. A kettő határa azonban nincs konkrétan megszabva, mert az
ráadásul még a forrás által sugárzott domináns hullámhossztól is függ.

Ha megvizsgáljuk a Hertz-dipólusra, mint elemi antennára felírható egyenleteket,
akkor sok hasznos tanulságot levonhatunk a közel- és távoltér különbségéről.

A gömbkoordináta rendszerben kifejezett $\mathbf{E}$ és $\mathbf{H}$ térerősségek:

\begin{equation}
H_{\phi} = j \frac{I \delta l}{4 \pi}
\left( \frac{k}{r} - \frac{i}{r^2} \right)
e^{-jkr} \sin \theta
\end{equation}

\begin{equation}
E_{\theta} = j \frac{Z_0 I \delta l}{4 \pi}
\left( \frac{k}{r} - \frac{i}{r^2} -\frac{1}{kr^3}\right)
e^{-jkr} \sin \theta
\end{equation}

\begin{equation}
E_{r} = j \frac{Z_0 I \delta l}{2 \pi}
\left( \frac{k}{r} - \frac{i}{kr^3}\right)
e^{-jkr} \cos \theta
\end{equation}
ahol a 
\begin{itemize}
\item $Z_0 = \sqrt{{\mu_0}/{\epsilon_0}}$ a hullámimpedancia,
\item $r, \theta$ és  $\phi$ a gömbkoordináta rendszer paraméterei,
\item $I$ az áramerősség,
\item $\delta l$ az antenna infinitezimálisan kicsi hossza
\end{itemize} 

A távoltérben az ${1}/{r^2}$-tel és ${1}/{r^3}$-bel arányos tagok
gyorsan csökkennek, és az ${1}/{r}$-rel arányos tag marad a domináns.

A távoltérben a sugárzás elektromos és mágneses térereje a 
távolsággal fordítottan arányos, 
míg a közeltérben még az ennél sokkal gyorsabban csökkenő, a távolság négyzetével illetve 
harmadik hatványával fordítottan arányosan tagok dominálnak.
Míg a távoltérben az elnyelődés nem hat vissza az adóra, addig a közeltérben a sugárzás
elnyelődése megváltoztatja az adó terhelését. Ennek a jelenségnek egy egyszerű
modellje a például transzformátorokban megfigyelhető mágneses indukció.

Az elektromágneses mezőben az elektromos és mágneses mező a távoltér esetén kapcsolatban
állnak és a kettejük intenzitásának arányát a hullámimpedancia adja meg. 
Ezzel ellentétben a közeltérben a két mező egymástól függetlenül létezhet, és
az egyik teljesen el tudja nyomni a másikat.

Egy antennában a pozitív és negatív töltések nem hagyhatják el az antennát és a 
gerjesztő jel által elválasztva, annak megfelelően oszcilláló elektromos dipólust
alkotnak, ami mind a közeltérben, mind a távoltérben hatást fejt ki. Az antennák
többnyire arra lettek kitalálva, hogy nagy távolságú vezetéknélküli kommunikációt
lehessen velük folytatni, és ezeknek az antennáknak ez a régió a normális működési
tartományuk. Természetesen a közeltér antennákat is egyre gyakrabban használjuk
adatátvitelre, hiszen az \emph{NFC} is ennek a képviselője.

A kettő régió között van az átmeneti zóna, amely az antenna geometriájától és a 
hullámhossz hosszától függ.

\subsection{Definíciók}
A közeltér-távoltér szóhasználatnál nagyon fontos, hogy tisztázzuk a pontos
definíciókat, hiszen az egyes felhasználási területektől függően különböző
jelentést hordozhatnak. Itt most az elektromágneses hosszúság szerinti bontás
definíciói fontosak számunkra.

\subsubsection{Elektromágnesesen rövid antennák}
Ezek azok az antennák, amelyek rövidebbek, mint az általuk kibocsájtott hullámhossz
hosszának a fele. Ezeknél az antennáknál a közeltér-távoltér különbségtételt egyszerűen
a sugárzó forrástól való távolság ($r$) és a hullámhossz ($\lambda$) aránya adja meg.
Az ilyen antennáknál közeltérnek hívjuk, ha $ r \ll \lambda $ és távoltérnek, ha $ r \gg 2\lambda $.
A kettő közötti távolságot átmeneti zónának nevezzük.

Érdemes megemlíteni, hogy az antenna hossza nem fontos, és ez a becslés működik minden
rövidebb antennára is (ideális esetekben pontantennáknak is hívják őket).

\subsubsection{Elektromágnesesen hosszú antennák}
Azoknál az antennáknál, amelyeknek a fizikai hossza nagyobb, mint az általuk sugárzott
hullámhossz fele, azoknál a közel- és távoltér régiók határának a meghatározására a 
Frauenhofer távolságot használják. Ennek értékét a:

\begin{equation}
d_f = \frac{2D^2}{\lambda}
\end{equation}

egyenlet adja meg, ahol a
\begin{itemize}
\item $D$ a legnagyobb mérete az antennának (pl. tányérantennáknál az átmérő),
\item $\lambda$ a rádióhullám hullámhossza.
\end{itemize} 

Az ilyen elektromágnesesen hosszúnak tekinthető antennák jelentősen megnövelik a
közeltér hatásokat. Másik szemszögből vizsgálva, ha egy adott antenna nagy 
frekvenciájú sugárzást bocsájt ki, akkor nagyobb lesz a közeltér régiója, ami a
rövidebb hullámhossznak köszönhető.

\section[NFC]{NFC, avagy Near Field Communication}
Az NFC technológia, mint ahogy a neve is mutatja a közeltér felhsználásával
megvalósított vezeték nélküli kommunikációs protokoll. Közvetlen elődjének az
RFID (Radio Frequency Identification) tekinthető, de sokban eltér tőle. A 
kettejük különbségeit és hasonlóságait egy későbbi fejezetben tárgyalom.
Az NFC olyan vezeték nélküli technológia, amely kis sávszélesség mellett
nagy frekvencián nyújt adatátvitelt, néhány centiméteres távolságban.

\subsection{Specifikáció összefoglalás}
\begin{itemize}
\item Az NFC mágneses mező indukálásával kommunikál kettő hurokantenna között,
amelyek egymás közelterében  helyezkednek el, ezzel egy levegőmagos
transzformátort hozva létre. A globálisan, engedély nélkül igénybe vehető 
13,56 MHz-en működik akár 1,8 Mhz-es sávszélességgel.
\item A működési távolságának a maximuma 10 cm körül alakul.
\item Támogatott átviteli adatsebességek: 106, 212 vagy 424 kbit/s.
\item Kétféle működési mód létezik:
\begin{itemize}
\item Passzív kommunikációs mód: A kezdeményező fél hozza létre a vivő mezőt, és a
cél eszköz ezt a mezőt modulálva válaszol. Ebben a módban a céleszköz a 
működéséhez szükséges energiát a kezdeményező által létrehozott elektromágneses
mezőből nyeri.
\item Aktív kommunikációs mód: Mind a kezdeményező, mind a céleszköz az általuk
felváltva keltett mezővel kommunikálnak. Amennyiben adatra vár az egyik eszköz,
annyiban kikapcsolja a saját mezejét. Ebben a működési módban mindkét eszköznek
rendelkeznie kell saját energiaellátással.
\end{itemize}
\item Az NFC kétféle kódolást használ adattovábbításra. Ha egy aktív eszköz 
106 kbit/s melletti adattovábbítást végez, akkor 100\%-os moduláció mellett
módosított Miller kódolást alkalmaz. Minden más esetben Manchester kódolást
használ 10\%-os modulációs arány mellett.
\item Az NFC-s eszközök képesek adatot egyszerre fogadni és továbbítani is. 
\end{itemize}

\subsection{Frekvenciasáv és sebesség}
Az NFC a 13,56-os ISM (Industrial, Scientific and Medical: ipari, tudományos
és orvosi) frakvenciát használja. A rádiófrekvenciás energia nagy része a 
$\pm 7$ kHz-es engedélyezett sávban koncentrálódik, de a teljes spektrális 
kiterjedés akár 1,8 Mhz is lehet ASK moduláció használata esetén.

A sebesség 106, 212 és 424 kbit/s-os sebessége alacsonynak tűnhet a mai egyéb
vezetéknélküli technológiákhoz képest, mint amilyen a Bluetooth, vagy a Wi-fi.
Ez is azt mutatja, hogy az adatátvitel sebessége nem volt elsődleges elsődleges
a tervezésnél. Ellenben nagyon kényelmesen és gyorsan képes a kapcsolat
felépítésére.

\subsection{Kommunikációs módok}
Az NFC interfész aktív vagy passzív módban tud működni. Az aktív eszköz saját
mezőt hoz létre, míg a passzív módban működő eszköznek az induktív csatolást
kell felhasználnia az adattovábbításra. Tehát a passzív módban működő eszközöknek
nem kell belső energiaforrással rendelkezniük, illetve az akkumulátorral működő
eszközöknek, mint amilyenek a mobiltelefonok, érdemes ezt a módot használniuk, 
amennyiben a másik oldal képes aktív módban működni. A passzív módban működő
eszköz az aktív eszköz mezejének terhelésmodulációját használja az
adattovábbításra. Így az eszköz alkalmas kártya emulációra, például 
jegykezelő rendszerekben való felhasználásra,akár  akkor is, amikor a mobiltelefon
ki van kapcsolva.

Összefoglalva tehát aktív a kommunikációs mód, ha két aktív eszköz kommunikál 
egymással. Mindig az az eszköz hozza létre a saját mezejét, amelyik adatot
szeretne küldeni. A két eszköz felváltva hozza létre a teret.

Passzív módnak azt nevezzük, amikor egy aktív eszköz és egy passzív eszköz között
jön létre a kapcsolat. Az aktív eszköz hozza létre a rádiófrekvenciás mezőt.

\subsection{Szerepek}
Kétféle szerepet lehet megkülönböztetni NFC kommunikáció során. Kezdeményezőnek 
(initiator) nevezzük azt, amelyik az első üzenetet küldi és célpontnak (target)
a fogadó felet. Passzív eszköz nem kezdeményezhet adatcserét. A célpont addig 
nem küldhet vissza választ, amíg a kezdeményező nem indítja el a kommunikációt.
Egy kezdeményező több célponttal is felvehet kapcsolatot.

\subsection{Összeütközés elkerülése}
Általában ritkák a félreértések, hiszen az eszközöket közvetlen egymás mellé 
szükséges helyezni. A protokoll egy egyszerű alapszabályt alkalmaz: hallgatás
közlés előtt (listen before talk). Ha a kezdeményező kommunikálni szeretne,
akkor biztosra kell mennie, hogy nincsen külső mező, amelyiket megzavarhat,
vagy amelyik megzavarhatja az NFC kommunikációt. Egy pontosan meghatározott 
ideig csöndben kell várakoznia a kezdeményezés előtt, miután idegen mezőt 
érzékelt. Ha kettő vagy több eszköz pontosan ugyanakkor válaszol, akkor a
kezdeményező érzékeli az összeütközést.

\subsection{Kódolások}

\subsubsection{Manchester kódolás}
A Manchester kódolásnál a jel értéke a periódus közepén létrejövő váltás irányától
függően vehet fel 0 vagy 1-es értéket. 0 bitet akkor vesz fel, amikor a jel
alacsony szintről magas szintre vált, 1 bitet akkor, amikor a magas jelszintről
alacsonyra vált. Ennek köszönhetően minden periódus közepén szintet vált a jel.
A periódus elején fellépő változások nem játszanak szerepet.

\subsubsection{Módosított Miller kódolás}
A módosított Miller kódolásnál az 1 és 0 bitek értelmezése attól függ, hogy
a jelben egy perióduson belül mikor megy végbe változás. Az egyes bitet
mindig ugyanúgy kódoljuk: a perióduson belül a jelszint a második félperiódusban
esik 0-szintre. A nullás bit kódolása függ attól, hogy milyen bit előzte meg.
Amennyiben egyes bit előzte meg, annyiban végig magas jelszint jelenti a nullás
bitet, amennyiben nulla előzte meg, annyiban egy olyan jelalak, amely az első
félperiódusban veszi fel az alacsony jelszintet.

\subsection{Biztonsági vonások}
Először is fontos 

\subsubsection{Lehallgatás}

\subsubsection{Adattörlés}

\subsubsection{Adatmódosítás}

\subsubsection{Adatbeszúrás}

\subsubsection{Man-in-the-Middle}

\section{Az NFC és egyéb vezetéknélküli technológiák}

\end{document}
