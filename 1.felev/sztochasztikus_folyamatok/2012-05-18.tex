\documentclass[a4paper,12pt]{article}
\usepackage[utf8]{inputenc}
\usepackage[T1]{fontenc}
\usepackage{lmodern}
\usepackage{indentfirst}
\usepackage[bookmarks, colorlinks=true, linkcolor=black, citecolor=black, urlcolor=blue]{hyperref}
\usepackage{amsmath}
\usepackage{hyperref}
\usepackage{hyphenat}
\usepackage[magyar]{babel}

\sloppy
\frenchspacing
\setlength{\parskip}{\smallskipamount}

\title{Sztochasztikus folyamatok 2015-06-06 vizsga}

\hyphenation{va-ló-szí-nű-ség-gel}

\begin{document}
\maketitle

Figyelem! Természetesen a kidolgozás elolvasása nem ment fel a készülés
alól. Csak azoknak tud érdemben segítséget nyújtani, akik egyébként is
készültek. A feladatok helyességére semmilyen garanciát nem vállalok!
Ha hibát találsz benne, akkor jelezd, vagy javítsd ki. A \LaTeX\ forrás és 
a \texttt{.pdf} elérhető az \url{https://github.com/dyj216/msc-sze-vill} 
címen. A fájlokat a ,,Raw'' gombra való kattintással lehet letölteni.

A felkészüléshez ajánlom a tanár úr honlapján megtalálható 
\texttt{\mbox{sztocha\_01.pdf}--\mbox{sztocha\_08.pdf}}-eket. Ezekre vannak hivatkozások 
ebben a kidolgozásban is. Ezek elérhetőek a 
\url{http://www.sze.hu/~harmati/sztocha2.html} címen. A vizsgákon a 
feladattípusok ugyanazt a sémát követik, minden feladat egy--egy 
anyagrészre kérdez rá, természetesen a pontos feladattípusok vizsgáról
vizsgára változnak.

\section*{1. feladat}
Egy bolha ugrál az $1,2,\dots,40$ számokon. Ha a $2,3,\dots39$ számok valamelyikén van, akkor $1/2$ valószínűséggel lép jobbra, $1/2$ valószínűséggel lép balra, de ha 1-ben vagy 40-ben van, ott is marad. (a falak elnyelőek).

a)	TFH 36-ból elindul. Mi a VSZ-e, hogy 1-ben végzi? (4p)

b)	TFH 14-ből indul. Várhatóan hány lépés múlva fejeződik be a bolyongása? (4p)

\subsection*{a)}
A bolyongás elnyelő falakkal megegyezik a tönkremenési problémával. Mivel 
$p = 1 - p = 1/2$, ezért $P = n/A$.

(Ebben nem vagyok biztos) Mivel most $1$ az elnyelő állapot, ezért $A=39$, 
és $n$ az a közelebbi elnyelő
állapottól való távolság: $n = 4$. Így a kérdésre a válasz:

\[
P(\textrm{a bolha 1-ben végzi}) = \frac{n}{A} = \frac{4}{39} \approx 0,1026 
\]

\subsection*{b)} 
A bolyongás várható időtartama $p = 1-p = 1/2$ esetén: $d(n) = n(A-n)$.
Most $n = 13$, $A = 39$.

\[
d(n) = 13 \cdot (39-13) = 338 
\]

\section*{2. feladat}
Egy Markov-lánc állapottere: $S = \{1, 2, 3, 4, 5\}$, átmenet-valószínűség mátrixa:

\[
P = 
\begin{bmatrix} 
0    &    1/4    &    1/4    &    1/4    &    1/4 \\
0    &    0      &    0      &    1/2    &    1/2 \\
0    &    0      &    0      &    1/2    &    1/2 \\
1/2    &    1/2      &    0      &    0    &    0 \\
1/2    &    0      &    0      &    0    &    1/2
\end{bmatrix}
\]

a)	Irreducibilis-e a Markov-lánc? (3p)

b)	Határozzuk meg az invariáns eloszlást! (5p)

c)	Tegyük fel, hogy 1-ből indulunk. Várhatóan hány lépés múlva érünk ismét 1-be? (3p)

\subsection*{a)}
Irreducibilis a Markov-lánc, hiszen egyetlen rekurrens osztályból áll, ami azt
jelenti, hogy bármely állapotból elérhető bármely másik állapot. Ez az egy 
rekurrens osztály az $\{1, 2, 3, 4, 5\}$.

\subsection*{b)}
Egyszerű invariáns eloszlás számolás, segítségképpen a mátrix, oldalain 
feltüntetve az állapotokkal:

\[
P = 
\bordermatrix{
~	&	1	&	2     &    3    &    4    &    5 	\cr
1	&	0	&	1/4   &    1/4  &    1/4  &    1/4  \cr
2	&	0	&	0     &    0    &    1/2  &    1/2  \cr
3	&	0	&	0     &    0    &    1/2  &    1/2  \cr
4	&	1/2	&	1/2   &    0    &    0    &    0    \cr
5	&	1/2	&	00    &    0    &    0    &    1/2}
\]

Az oszlopokba kell belenézni az invariáns eloszlás egyenletrendszerének
felírásához:

\begin{align*}
\pi_1 &= 1/2 \pi_4 + 1/2 \pi_5 \\
\pi_2 &= 1/4 \pi_1 + 1/2 \pi_4 \\
\pi_3 &= 1/4 \pi_1             \\
\pi_4 &= 1/4 \pi_1 + 1/2 \pi_2 + 1/2 \pi_3 \\
\pi_5 &= 1/4 \pi_1 + 1/2 \pi_2 + 1/2 \pi_3 + 1/2 \pi_5 \\
\sum_i \pi_i = 1 &= \pi_1 + \pi_2 + \pi_3 + \pi_4 + \pi_5
\end{align*}

Megoldva az egyenletrendszert (természetesen lehet egyszerűsíteni, 
ilyen formában viszont jól látszik, hogy az összegük 1-et ad):

\[
\pi_1 = \frac{12}{46}, \quad
\pi_2 = \frac{7}{46}, \quad
\pi_3 = \frac{3}{46}, \quad
\pi_4 = \frac{8}{46}, \quad
\pi_5 = \frac{16}{46}
\]

\subsection*{c)}
Az invariáns eloszlás ismeretében ez könnyű:

\[
E(T_{1 \to 1}) = \frac{1}{\pi_1} = \frac{46}{12} = \frac{23}{6}
\]

\section*{3. feladat}
Két dobozban ($A$ és $B$) 3-3 golyó van, összesen 3 fehér és 3 piros. 
Egy lépés abból fog állni, hogy $A$-ból és $B$-ből is kihúzunk egy-egy 
golyót, és kicseréljük őket. Tekintsük a piros golyók számát az $A$
dobozban.

a)	Írjuk fel az átmenet-valószínűség mátrixot! (5p)

b)	Hosszú távon az idő hányad részében nem lesz piros golyó az $A$-ban? (5p)

c)	Tegyük fel, hogy kezdetben minden piros golyó $A$-ban volt. 
Várhatóan hány lépés múlva lesz benne csak fehér? (5p)

\subsection*{a)}
Az piros golyók száma az $A$ dobozban az alábbi értékeket vehetik fel:
$\{0, 1, 2, 3\}$. Ennek megfelelően az átmenet-valószínűségmátrix:

\[
P = 
\bordermatrix{
~	&	0    &    1    &    2    &    3    \cr
0	&	0    &    1    &    0    &    0    \cr
1	&	1/3 \cdot 1/3 = 1/9    &    4/9    &    2/3 \cdot 2/3 = 4/9    &    0    \cr
2	&	0    &    4/9    &    4/9    &    1/9    \cr
3	&	0    &    0    &    1    &    0    }
\]

Ha 1 golyó piros $A$-ban, akkor úgy lehet, benne 0, ha a 3 golyóból 
egyedüli pirosat kivesszük, és kicseréljük a 3 golyóból 1 fehérrel 
a $B$-ből. Ahhoz, hogy kettő piros golyó legyen az $A$-ban, ahhoz a 3 
golyóból a kettő fehér közül az egyiket kell kihúzni, aminek $2/3$-ad 
a valószínűsége, és a $B$-ből a 3 golyó közül a 2 piros egyikét kell
kihúzni, ennek a valószínűsége is $2/3$, így kapjuk a kettő szorzatából
a $4/9$-es valószínűséget. Az $1 \to 1$ átmenethez 1-ből kivonjuk az 
adott sorban lévő többi valószínűséget (vagy végiggondoljuk).

\subsection*{b)}
Ehhez invariáns eloszlást kell számolni. A keresett érték:

\[P(\textrm{nincs a 0 állapotban}) = 1 - \pi_0\]

Az invariáns eloszláshoz az egyenletrendszer:

\begin{align*}
\pi_0 &= 1/9 \pi_1 \\
\pi_1 &= \pi_0 + 4/9 \pi_1 + 4/9 \pi_2 \\
\pi_2 &= 4/9 \pi_1 + 4/9 \pi_2 + \pi_3             \\
\pi_3 &= 1/9 \pi_2 \\
\sum_i \pi_i = 1 &= \pi_0 + \pi_1 + \pi_2 + \pi_3
\end{align*}

\section*{4. feladat}
Egy boltba Poisson-folyamat szerint érkeznek a vevők, átlagosan két 
percenként egy.

a)	Mi a valószínűsége, hogy 10 perc alatt pont 4 vevő érkezik? (3p)

b)	Öt percig nem jött senki. Mi a VSZ-e, hogy a következő percben 
pont két vevő jön? (3p)

c)	Tegyük fel, hogy 10 perc alatt 8-an jöttek. Mi a valószínűsége, 
hogy az első 2 percben nem jött senki? (4p)

\section*{5. feladat}
Egy hivatalnál egyetlen ablaknál intézik az ügyeket. A kiszolgálási 
idő exponenciális eloszlású 10 perc várható értékkel. Az ügyfelek 
Poisson-folyamat szerint érkeznek, óránként átlagosan 5. Tegyük fel, hogy 
a hivatal befogadóképessége végtelen.

a)	Mi a valószínűsége, hogy legalább 2 ügyfél van a hivatalban? (4p)

b)	Átlagosan hányan várakoznak a hivatalban?(4p)

c)	Átlagosan mennyi időt tölt el egy ügyfél a hivatalban? (4p)

d)	Mennyi az átlagos várakozási idő? (4p)


\end{document}