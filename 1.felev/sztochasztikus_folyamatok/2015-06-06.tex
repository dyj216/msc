\documentclass[a4paper,12pt]{article}
\usepackage[utf8]{inputenc}
\usepackage[T1]{fontenc}
\usepackage{lmodern}
\usepackage{indentfirst}
\usepackage[bookmarks, colorlinks=true, linkcolor=black, citecolor=black, urlcolor=blue]{hyperref}
\usepackage{amsmath}
\usepackage{hyperref}
\usepackage{hyphenat}
\usepackage[magyar]{babel}

\sloppy
\frenchspacing
\setlength{\parskip}{\smallskipamount}

\title{Sztochasztikus folyamatok 2015-06-06 vizsga}

\hyphenation{va-ló-szí-nű-ség-gel}

\begin{document}
\maketitle

Figyelem! Természetesen a kidolgozás elolvasása nem ment fel a készülés
alól. Csak azoknak tud érdemben segítséget nyújtani, akik egyébként is
készültek. A feladatok helyességére semmilyen garanciát nem vállalok!
Ha hibát találsz benne, akkor jelezd, vagy javítsd ki. A \LaTeX\ forrás és 
a \texttt{.pdf} elérhető az \url{https://github.com/dyj216/msc-sze-vill} 
címen. A fájlokat a ,,Raw'' gombra való kattintással lehet letölteni.

A felkészüléshez ajánlom a tanár úr honlapján megtalálható 
\texttt{\mbox{sztocha\_01.pdf}--\mbox{sztocha\_08.pdf}}-eket. Ezekre vannak hivatkozások 
ebben a kidolgozásban is. Ezek elérhetőek a 
\url{http://www.sze.hu/~harmati/sztocha2.html} címen. A vizsgákon a 
feladattípusok ugyanazt a sémát követik, minden feladat egy--egy 
anyagrészre kérdez rá, természetesen a pontos feladattípusok vizsgáról
vizsgára változnak.

Ezt a vizsgasort emlékezetből írtam meg, a feladatok megfogalmazása
biztosan nem pontos.

\section*{1. feladat}
Szerencsejátékot játszok, 0,49 valószínűséggel nyerem el ellenfelem 
pénzét, ő 0,51-gyel nyeri el az én pénzemet. 200 Ft-om van,
ellenfelemnek 40.

(a) Milyen tét megválasztásával tudom maximalizálni a nyerési 
esélyemet?

(b) Mennyi a valószínűsége annak, hogy nyerek ilyen tétek mellett, 
mennyi annak, hogy vesztek?

(c) Ha ellenfelemnek megengedem, hogy ő választhasson tétet, 
akkor ő mekkora tétet fog választani?

(d) Ha végtelen pénzem van, meddig növelhetem a nyerési esélyemet?

\subsection*{a)}
Ha a játék nem nekünk kedvez (de nekünk van több pénzünk), akkor
a lehető legnagyobb tét kiválasztásával tudjuk a nyerési esélyünket
maximalizálni. Ez 40 Ft.

\subsection*{b)}
Annak az esélye, hogy nyerek 40 Ft-os tét mellett 
($n_1 = 200/40 = 5, \ n_2 = 40/40 = 1$):

\[
P(\textrm{nyerek}) = 
\frac{1 - \left( \frac{1-p}{p} \right)^{n_1}}
{1 - \left( \frac{1-p}{p} \right)^{n_1 + n_2}} = 
\frac{1 - \left( \frac{0,51}{0,49} \right)^{5}}
{1 - \left( \frac{0,51}{0,49} \right)^{6}} \approx 0,8162
\]

Annak a valószínűsége, hogy vesztünk:

\[
P(\textrm{vesztek}) = 1 - P(\textrm{nyerek}) = 0,1838
\]

\subsection*{c)}
Ellenfelemnek kedvez a játék, így neki a lehető legkisebb tétet kell
választania, hogy maximalizálja a nyerési esélyét. 
Ekkor $n_1 = 200$, $n_2 = 40$, és az én nyerési esélyem:

\[
P(\textrm{nyerek}) = 
\frac{1 - \left( \frac{1-p}{p} \right)^{n_1}}
{1 - \left( \frac{1-p}{p} \right)^{n_1 + n_2}} = 
\frac{1 - \left( \frac{0,51}{0,49} \right)^{200}}
{1 - \left( \frac{0,51}{0,49} \right)^{240}} \approx 0,2018
\]

\subsection*{d)}
1 kört inkább elvesztünk, mint megnyerünk ($p < q$), ellenfelünk pénze
konstans ($n_2 = 5$), így alkalmazhatjuk a következő határérték számítást:

\[
\begin{split}
P(\text{nyerünk}) &= 
\lim_{n_1 \to \infty} \frac{1-\left(\frac{1-p}{p}\right)^{n_1}}
{1-\left(\frac{1-p}{p}\right)^{n_1+n_2}} =
\lim_{n_1 \to \infty} \frac{1-\left(\frac{1-p}{p}\right)^{n_1}}
{1-\left(\frac{1-p}{p}\right)^{n_1}\left(\frac{1-p}{p}\right)^{n_2}} =\\
&= \lim_{n_1 \to \infty} \frac{\left(\frac{1-p}{p}\right)^{n_1}}
{\left(\frac{1-p}{p}\right)^{n_1}} \cdot 
\frac{\frac{1}{\left(\frac{1-p}{p}\right)^{n_1}}-1}
{\frac{1}{\left(\frac{1-p}{p}\right)^{n_1}} -
\left(\frac{1-p}{p}\right)^{n_2}} 
= \frac{-1}{-\left(\frac{1-p}{p}\right)^{n_2}} =\\
&= \left(\frac{p}{1-p}\right)^{n_2} = 
\left(\frac{0,49}{0,51}\right)^{40} \approx 0,2019
\end{split}
\]

\section*{2.feladat}
Adott az $\{1, 2\}$ állapotokkal rendelkező Markov-lánc 
az alábbi állapotvalószínűségi mátrixszal:

\[
P = 
\begin{bmatrix} 
0,2    &    0,8 \\
0,4    &    0,6
\end{bmatrix}
\]

(a) Adja meg a $P(X_1 = 1 \mid X_0 = 2)$ valószínűséget!

(b) Adja meg a $P(X_4 = 2 \mid X_2 = 2)$ valószínűséget!

(c) Adja meg az invariáns eloszlást!

\subsection*{Bevezető}
Az (a) és (b) jellegű kérdésekre borzasztóan egyszerű a válasz:
meg kell nézni, hogy a keresett valószínűség az  melyik időpillanatban 
(lépésben) van, ezt $X$ alsó indexéből leolvashatjuk ((a) esetben ez 1,
(b) esetben ez 4). 
Ebből le kell vonni a feltétel időpillanatának (lépésének) az értékét 
((a) esetben ez 0, (b) esetben ez 2), tehát $1-0=1$ és $4-2=2$, és a 
$P$ mátrix ennyiedik hatványát kell megnézni
((a) esetben $P^1$-et, (b) esetben $P^2$-t). A sorok jelölik az, hogy 
melyik állapotban vagyunk, az oszlopok, hogy melyik állapotba megyünk.

\subsection*{a)}
Tekintsük tehát a $P^1$ mátrixot:

\[
P = 
\bordermatrix{
~	&	1	&	2	\cr
1	&	0,2	&	0,8	\cr
2	&	0,4	&	0,6}
\]

A keresett valószínűség ennek a mátrixnak az első sorának második eleme.

\[
P(X_1 = 1 \mid X_0 = 2) = 0,4
\]

\subsection*{b)}
Mivel itt a különbség a két időpont között 2, ezért ki kell számolni a 
$P^2$ mátrixot. Akinek a számológépe tudja ezt az csak bepötyögi és 
lemásolja az eredményt,  többiek kézzel kiszámolják.

\[
P^2 = 
\bordermatrix{
~	&	1	    &	2    	\cr
1	&	0,36	&	0,64	\cr
2	&	0,32	&	0,68}
\]

Innen a keresett valószínűség az ennek a mátrixnak a második sorának
második eleme.

\[
P(X_4 = 2 \mid X_2 = 2) = 0,68
\]

\subsection*{c)}
Ez a rész egy sima invariáns eloszlás (jelölése: $\underline{\pi}$, 
mert egy vektor) számolás. Az invariáns eloszlás 
az azt jelenti, hogy ha $P$ kitevője már elég magas, akkor a kezdeti
eloszlás elveszti jelentőségét ($\varphi_0$-val szoktuk jelölni, ebben a
feladatban nem szerepel). Tehát:

\[
\lim_{n\to \infty} \varphi_0 P^n = \underline{\pi}
\] 

Ebből következik, hogy a magas kitevőjű $P$-t $\underline{\pi}$-vel
megszorozva $\underline{\pi}$-t kapunk:

\[
\underline{\pi} = \lim_{n\to \infty} \varphi_0 P^n = 
\lim_{n\to \infty} \varphi_0 P^{n+1} = 
\lim_{n\to \infty} \varphi_0 P^n P = \underline{\pi} P
\]

\[
\underline{\pi} = \underline{\pi} P
\]

Invariáns eloszlást a mátrix oszlopaiból felírt
egyenletrendszerrel lehet számolni. $\pi_1$ az egyes állapot invariáns
eloszlása $\pi_2$ a kettes állapoté. Ez az egyenletrendszer, tudva, hogy 
az invariáns eloszlások összege 1-et ad eredményül:


\begin{eqnarray*}
\pi_1 = 0,2 \pi_1 + 0,4 \pi_2 \\
\pi_2 = 0,8 \pi_1 + 0,6 \pi_2 \\
\sum_i \pi_i = 1 = \pi_1 + \pi_2
\end{eqnarray*}

Ezt az egyenletrendszert megoldva kapjuk az invariáns eloszlás 
értékeket, amit mátrix számolás képes számológéppel könnyen
ellenőrizhetünk, úgy, hogy a mátrixunkat sokszor 
(mondjuk 5--10-szer négyzetre emeljük):

\[
\pi_1 = \frac{1}{3},\quad \pi_2 = \frac{2}{3}
\]

\section*{3. feladat}
Adott két doboz $A$ és $B$, és 8 golyó: 6 fehér és 2 sárga. Egy lépés
jelentse azt, hogy megfogunk $A$-ban és $B$-ben is egy--egy golyót és
kicseréljük őket. Jelölje az állapotokat az $A$-ban lévő sárga golyók
száma.

(a) Írja fel az állapotvalószínűségi mátrixot!

(b) ??? Talán az volt, hogy adja meg az invariáns eloszlást ???

\subsection*{a)}
Ha megértjük, hogy miket jelentenek az állapotok, ez a feladat is
könnyű lesz. Most az állapotaink lehetnek $\{0, 1, 2\}$, hiszen
ennyi sárga golyó lehet maximálisan az $A$-ban. A mátrix:

\[
P = 
\bordermatrix{
~	&	0    &    1      &	    2	\cr
0	&	1/2  &    1/2    &	    0	\cr
1	
&	(1/4 \cdot 3/4 = 3/16)
&	10/16
&   (1/4 \cdot 3/4 = 3/16)   \cr
2	&	0    &    1/2    &      1/2}
\]

Egy kis magyarázat: 
\begin{itemize}
\item Amikor 0 db sárga golyó van az $A$-ban, akkor
1 a valószínűsége annak, hogy az $A$-ból fehéret veszünk ki, és
$1/2$ annak a valószínűsége, hogy a $B$-ből fehéret veszünk ki, 
ezek szorzata adja a mátrix $(0, 0)$ elemét. 
\item Szintén $1/2$ annak a valószínűsége,
hogy sárgát veszünk ki a $B$-ből, ez a mátrix $(0, 1)$-es elemét adja.
\item Az $(1, 0)$-ás elemet úgy kapjuk, hogy ha már van egy sárga golyó
az $A$ dobozban, akkor akkor $1/4$ annak a valószínűsége, hogy azt
húzzuk ki és tesszük át a $B$-be, és akkor lesz 0 sárga golyó az 
$A$-ban, ha a $B$-ből fehér golyót teszünk át $A$-ba, aminek $3/4$ a 
valószínűsége. Ennek a kettőnek a szorzata adja a $3/16$-ot.
\item Az $(1, 2)$-es elemre is hasonlóan juthatunk, mint ahogy az
előző pontban jutottunk.
\item Az $(1, 1)$-es elemet szintén megkaphatjuk, ha végiggondoljuk, 
hogy hogyan lehet az, hogy 1 sárga golyó volt az $A$-ban és 1 lesz a
csere után is, de könnyebb onnan kiszámolni, hogy a sorok összegének
1-et kell adnia a Markov--láncokban, tehát $1-3/16-3/16 = 10/16$ ennek
a valószínűsége.
\end{itemize}

\section*{4. feladat}
Erre már nem emlékszem, csak annyira, hogy tipikus Poisson--eloszlású
feladat volt, az (a) részben egy időtartam alatt érkezők száma volt a
kérdés, a (b) részben egy független feltétel volt vizsgálva, azaz 
elhagyható volt az a feltétel, és a (c) részben pedig egy feltételes
valószínűséget kellett számolni.

\section*{5. feladat}
Erre sem emlékszem, de ez meg egy M/M/1-es típusú feladat volt.

\end{document}
